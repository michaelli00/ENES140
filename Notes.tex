\documentclass{article}
\usepackage[top=1in,bottom=1in]{geometry}
\usepackage{hyperref}
\usepackage{amsmath}
\usepackage{amssymb}
\usepackage{graphicx}
\graphicspath{ {./assets/} }
\usepackage[none]{hyphenat}
\date{}
\title{ENES140 Discovering New Ventures}
\begin{document} 
  \author{Michael Li}
  \title{ENES140 Discovering New Ventures}
  \maketitle
  \tableofcontents
  \newpage
  \section{Introduction}
    \textbf{Opportunity}: potential to create a new venture. \\
    \textbf{Business model}: engage entrepreneurs in customer discovery early on by putting emphasis on testing their hypotheses before investing effort into a business plan. However, this model relies on already having a product or service concept to test. \\
    \textbf{Opportunity Analysis Canvas}: before drafting business models, entrepreneurs need to view problems and solutions differently from others. Segmented into 9 steps:
    \begin{enumerate}
      \item Entrepreneurial mindset
      \item Entrepreneurial motivation
      \item Entrepreneurial behavior
      \item Industry condition
      \item Industry status
      \item Macroeconomic change
      \item Competition
      \item Value Innovation
      \item Opportunity Identification
    \end{enumerate}
    Steps can be broken down into:
    \begin{itemize}
      \item \textbf{Thinking entrepreneurially}: influenced by individual mindsets, motivations, and behaviors
      \item \textbf{Seeing entrepreneurially}: looking at the "big picture" perspective (e.g. economic forces at play)
      \item \textbf{Acting entrepreneurially}: transform ideas into actions
    \end{itemize}
  \section{What are entrepreneurial opportunities}
    \textbf{Entrepreneurial opportunities}: situations in which new goods and services are introduced and sold at a higher price than their production. Require the discovery of new relationships and interactions in the market that are uncertain and dynamic. Can be broken down into
    \begin{enumerate}
      \item Creating new information
      \item Exploiting market inefficiencies
      \item Reaction to shifts in relative costs and benefits of alternative resources
    \end{enumerate} 
    Today we see entrepreneurs influencing on a global scale (e.g. you can hire workers or products from across the globe). However this influence also brings competition since these collaborators are available to competitors as well.
    \section{What is Strategic Decision Making}
    Typical decisions involve
    \begin{itemize}
      \item Identifying a problem
      \item Generating alternatives
      \item Evaluating alternatives
      \item Selecting the best alternative based on criteria
    \end{itemize}
    Characteristics of strategic decision:
    \begin{itemize}
      \item \textbf{Complexity}: what makes a decision complex (variables and alternative choices)
      \item \textbf{Uncertainty}: not knowing the eventual outcome (can only anticipate)
      \item \textbf{Rationality}: identifying goals and objectives
      \item \textbf{Control}: what factors are controllable.
    \end{itemize}
    Rational decision makers base decision on data and logical analysis. They do not rely on their gut. \\
    Intuition base decision on feelings and emotions, avoiding the facts \\ \\
    \textbf{Concept development}: involves both market research and gut feeling (what benefits will your product bring forth?). \\
    \textbf{Market analysis}: use analytical analysis to identify if there is a customer out there and if there is any competitive force at play? Also involves intuition when deciding where the market and your competitors are heading. \\
    \textbf{Customer discovery}: need to do research on customer and what they want (rational) as well as act upon some intuition for a product. \\
    \textbf{Product design and prototyping}: need to test different alternatives (rational) while making some assumptions about what is the best option (intuitive)
    \section{Entrepreneurial Mindset}
    Entrepreneurs usually focus on 5 characteristics:
    \begin{itemize}
      \item Achievement: personal drive for accomplishments. Typically have goals set and reinforce these goals.
      \item Individualism: need less support or approval from others and place a high emphasis on independence, in contrast to collectivism.
      \item Control: split into:
        \begin{itemize}
          \item \textbf{autonomy} (you are the biggest influencer on decisions). Entrepreneurship is NOT a total autonomy; bosses are replaced with customers, partners, and investors.
          \item \textbf{locus of control}: individual's belief that they can influence others/environment, as opposed to \textbf{externally oriented} where individual is subject to other people and have minimal influence.
        \end{itemize}
        
      \item Focus: able to concentrate on a specific task.
      \item Optimism: being able to anticipate the best possible result. Typically measured by how other people see you. Can be learned through 5 steps:
        \begin{enumerate}
          \item Identify situation and document challenges
          \item Write beliefs and assumptions, trying to separate facts from feelings
          \item Think about consequences
          \item Think about other elements or what else may have happened
          \item Think about the level of influence of energy and beliefs over time, looking at how emotions have or could have influenced our decisions.
        \end{enumerate}
        Optimism should be combined with reality testing to avoid both ends of the spectrums.
    \end{itemize}
  \section{Entrepreneurial Motivation}
  Factors by which goal-directed behavior is initiated, energized, and maintained.
  \begin{itemize}
    \item \textbf{Self-efficacy}: belief in your ability to complete a specific task. Differs from \textbf{confidence} (the general case) whereas self-efficacy is task dependent.
    \item \textbf{Cognitive motivation}: (left-brain) individuals with this trait tend to seek, acquire, and analyze information and use logic to solve problem. (Righ-brain) individuals without this trait tend to rely on experience, intuition, and luck and think holistically (big picture).
    \item \textbf{Tolerance for Ambiguity}: tendency to perceive ambiguous or unclear situations as acceptable. It's a key trait because the market, customers, and competition are always changing so entrepreneurs need to be comfortable with the unexpected to make complex decision quickly with limited information.
  \end{itemize}
\end{document}
