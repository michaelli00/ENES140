\documentclass{article}
\usepackage[top=1in,bottom=1in]{geometry}
\usepackage{hyperref}
\usepackage{amsmath}
\usepackage{amssymb}
\usepackage{graphicx}
\graphicspath{ {./assets/} }
\usepackage[none]{hyphenat}
\date{}
\title{ENES140 Discovering New Ventures}
\begin{document} 
  \author{Michael Li}
  \title{ENES140 Discovering New Ventures}
  \maketitle
  \tableofcontents
  \newpage
  \section{Introduction}
    \textbf{Opportunity}: potential to create a new venture. \\
    \textbf{Business model}: engage entrepreneurs in customer discovery early on by putting emphasis on testing their hypotheses before investing effort into a business plan. However, this model relies on already having a product or service concept to test. \\
    \textbf{Opportunity Analysis Canvas}: before drafting business models, entrepreneurs need to view problems and solutions differently from others. Segmented into 9 steps:
    \begin{enumerate}
      \item Entrepreneurial mindset
      \item Entrepreneurial motivation
      \item Entrepreneurial behavior
      \item Industry condition
      \item Industry status
      \item Macroeconomic change
      \item Competition
      \item Value Innovation
      \item Opportunity Identification
    \end{enumerate}
    Steps can be broken down into:
    \begin{itemize}
      \item \textbf{Thinking entrepreneurially}: influenced by individual mindsets, motivations, and behaviors
      \item \textbf{Seeing entrepreneurially}: looking at the "big picture" perspective (e.g. economic forces at play)
      \item \textbf{Acting entrepreneurially}: transform ideas into actions
    \end{itemize}
  \section{What are entrepreneurial opportunities}
    \textbf{Entrepreneurial opportunities}: situations in which new goods and services are introduced and sold at a higher price than their production. Require the discovery of new relationships and interactions in the market that are uncertain and dynamic. Can be broken down into
    \begin{enumerate}
      \item Creating new information
      \item Exploiting market inefficiencies
      \item Reaction to shifts in relative costs and benefits of alternative resources
    \end{enumerate} 
    Today we see entrepreneurs influencing on a global scale (e.g. you can hire workers or products from across the globe). However this influence also brings competition since these collaborators are available to competitors as well.
    \section{What is Strategic Decision Making}
    Typical decisions involve
    \begin{itemize}
      \item Identifying a problem
      \item Generating alternatives
      \item Evaluating alternatives
      \item Selecting the best alternative based on criteria
    \end{itemize}
    Characteristics of strategic decision:
    \begin{itemize}
      \item \textbf{Complexity}: what makes a decision complex (variables and alternative choices)
      \item \textbf{Uncertainty}: not knowing the eventual outcome (can only anticipate)
      \item \textbf{Rationality}: identifying goals and objectives
      \item \textbf{Control}: what factors are controllable.
    \end{itemize}
    Rational decision makers base decision on data and logical analysis. They do not rely on their gut. \\
    Intuition base decision on feelings and emotions, avoiding the facts \\ \\
    \textbf{Concept development}: involves both market research and gut feeling (what benefits will your product bring forth?). \\
    \textbf{Market analysis}: use analytical analysis to identify if there is a customer out there and if there is any competitive force at play? Also involves intuition when deciding where the market and your competitors are heading. \\
    \textbf{Customer discovery}: need to do research on customer and what they want (rational) as well as act upon some intuition for a product. \\
    \textbf{Product design and prototyping}: need to test different alternatives (rational) while making some assumptions about what is the best option (intuitive)
    \section{Entrepreneurial Mindset}
    Entrepreneurs usually focus on 5 characteristics:
    \begin{itemize}
      \item Achievement: personal drive for accomplishments. Typically have goals set and reinforce these goals.
      \item Individualism: need less support or approval from others and place a high emphasis on independence, in contrast to collectivism.
      \item Control: split into:
        \begin{itemize}
          \item \textbf{autonomy} (you are the biggest influencer on decisions). Entrepreneurship is NOT a total autonomy; bosses are replaced with customers, partners, and investors.
          \item \textbf{locus of control}: individual's belief that they can influence others/environment, as opposed to \textbf{externally oriented} where individual is subject to other people and have minimal influence.
        \end{itemize}
        
      \item Focus: able to concentrate on a specific task.
      \item Optimism: being able to anticipate the best possible result. Typically measured by how other people see you. Can be learned through 5 steps:
        \begin{enumerate}
          \item Identify situation and document challenges
          \item Write beliefs and assumptions, trying to separate facts from feelings
          \item Think about consequences
          \item Think about other elements or what else may have happened
          \item Think about the level of influence of energy and beliefs over time, looking at how emotions have or could have influenced our decisions.
        \end{enumerate}
        Optimism should be combined with reality testing to avoid both ends of the spectrums.
    \end{itemize}
  \section{Entrepreneurial Motivation}
  Factors by which goal-directed behavior is initiated, energized, and maintained.
  \begin{itemize}
    \item \textbf{Self-efficacy}: belief in your ability to complete a specific task. Differs from \textbf{confidence} (the general case) whereas self-efficacy is task dependent.
    \item \textbf{Cognitive motivation}: (left-brain) individuals with this trait tend to seek, acquire, and analyze information and use logic to solve problem. (Righ-brain) individuals without this trait tend to rely on experience, intuition, and luck and think holistically (big picture).
    \item \textbf{Tolerance for Ambiguity}: tendency to perceive ambiguous or unclear situations as acceptable. It's a key trait because the market, customers, and competition are always changing so entrepreneurs need to be comfortable with the unexpected to make complex decision quickly with limited information.
  \end{itemize}
  \section{Entrepreneurial Behavior}
  4 critical behaviors to consider:
  \begin{itemize}
    \item \textbf{Confidence}: belief in one's powers or abilities. Related to self-confidence, self-reliance, and self-assurance. Confidence differs from self-efficacy in that self-efficacy only refers to an individual activity and confidence is a broader measure. \\ \\
      Overt examples of low self-confidence include: shy body language, hesitancy to speak up, and avoiding interaction. Internal signs include: indecisiveness, fear of failure, resisting trust, and seeking external validation. \\ \\
      Entrepreneurs need to have confidence in their own decisions, even when it is unpopular. Entrepreneurial opportunity discovery is limited if theres a lack of confidence in making decisions.
    \item \textbf{Risk}: the potential for loss (e.g. money or time). The decision to not start a venture is a decision that involves risk of lost profile or success. \\ \\
      There is no difference in risk tolerance between entrepreneurs and non-entrepreneurs. However, what we see as risky differs from person to person. \\ \\
      Most startups with funding from venture capitalists are more likely to survive than those without funding since they have sufficient financial capital to launch their ventures. Furthermore, funded companies use buy-ins; investors think your product is worth funding and have put money into your venture. \\ \\
      Venture capitalists usually avoid restaurants; the upfront costs of leasing, hiring a staff, and buying food before the restaurant actually opens is too costly. \\ \\
      Venture capitalists also avoid retail stores because of their limit uniqueness and low barriers to entry; other stores are usually selling the same products so uniqueness is limited by location and additional services provided. Furthermore, it's difficult to compete against online competitors. \\ \\
      Entrepreneurs can reduce risk by:
      \begin{itemize}
        \item \textbf{Search for information}: most entrepreneurs start doing things before planning. Thinking of what the market will look like in 6 months is a good start or what people will want to buy in 6 months.
        \item \textbf{Minimizing investments}: build a minimum viable product (MVP) to get feedback from customers. This feedback will help determine what you should focus working on. Repeat for later renditions of the product. Idea is to start with something simple, get it to market, get feedback, and repeat.
        \item \textbf{Maximize flexibility}: allows you to make pivots and adaptations to the product at a lower cost. Also lets you test different market segments and gives an opportunity to change the product if needed. \\ \\
        Startups should begin working on validating a concept that they will test with customers to get feedback.
      \end{itemize}
      Important thing to note is that entrepreneurs are not risk seekers, rather they see risk differently from non-entrepreneurs.
    \item \textbf{Interpersonal skills}: since entrepreneurs are working with team members, investors, customers all the time, interpersonal skills are important. Questions to assess interpersonal skills include:
      \begin{itemize}
        \item Do we isolate ourselves?
        \item Do we have difficulty expressing our feelings?
        \item Do we feel that others take advantage of us?
        \item Do we try and guess how we should act within a group?
        \item Do we take relationships seriously?
        \item Do we have problems developing intimate relationships?
        \item Do we ever feel guilty?
      \end{itemize}
      Interpersonal skills include building relationships with new people and maintaining existing relationships. This requires
      \begin{itemize}
        \item \textbf{Communication skills}: listening and responding thoughtfully is important. Focus on understanding on what's being said and respond accordingly.
        \item \textbf{Assertiveness}: be direct and specific in asking what you want.
        \item \textbf{Conflict resolution}: focus on the problem, not the individual, and seek discuss alternative solutions and next steps.
        \item \textbf{Anger management}: understand what is angering you and do not displace anger.
      \end{itemize}
      Interpersonal skills will improve the quantity and quality of relationships. It will also improve your ability to access entrepreneurial opportunities (other people know and like you). \\ \\
      Building relationships with stakeholders will open opportunities and give you access to information and resources.
    \item \textbf{Social capital}: gives you access and relationships that can't be bought. A diverse social capital (network) improves startup success. Social capital also works in multiple degrees: someone I know and someone they know. \\ \\
      One benefit of having rich social capital, is that you have more people to ask for help or ideas, ultimately saving time and money. Also social capital can lead to more funding.
  \end{itemize}
  \section{Industry Condition}
  Rules of competition within an industry, helping entrepreneurs decide what industries they want to join and avoid. 2 core segments that provide insights into how to compete effectively within an industry:
  \begin{itemize}
    \item \textbf{Knowledge conditions}: the quantity and quality of knowledge required to create industry products (e.g. what is the relative influence of human expertise)
      \begin{itemize}
        \item Industries with high knowledge conditions are intellectually more difficult to enter (higher barrier of entry)
        \item Industries with low knowledge conditions would be those anyone could enter if they had enough money, viable location, or the right relationships
      \end{itemize}
    \item \textbf{Demand conditions}: addressing the size of the market, its growth, and its consistency (need to satisfy customer needs in a better, profitable way). We can do this by, understanding the current needs and wants in the market and the future market, and understanding how competitors are responding. \textbf{Customer discovery} - going out and examining prospective customers - is critical for this.
  \end{itemize}
  \subsection{How to build knowledge}
  There are multiple ways of building knowledge at a startup
  \begin{itemize}
    \item Be knowledgeable on the subject yourself
    \item Find a co-founder or employ someone who is knowledgeable in the field
    \item Partner or joint venture with another company
    \item Outsourcing is used for short-term and project-based tasks (e.g. hire a small company to do a technical task)
  \end{itemize}
  \section{Industry Status}
  Studying industry status allows entrepreneurs to assess an industry's timeline for new entrants. Can be broken down into
  \begin{itemize}
    \item \textbf{Industry lifecycle}: life of the industry. Entrepreneurs should focus on being an early entrant into an industry to maximize chances for success.
    \item \textbf{Industry structure}: can be broken down into assessing
      \begin{itemize}
        \item \textbf{capital intensity}: amount of money required to enter and compete within an industry. Low-capital opportunities are ideal since then entry cost is low.
        \item \textbf{advertising intensity}: importance of advertising and branding to the success of competitors. Industries with high advertising intensity are those where customers prefer to buy based on past successful transactions or brands they know. Industries with low advertising intensity typically have customers trying out new things (this is where entrepreneurs should look to operate in).
        \item \textbf{company concentration}: ideally, we want a small number of competitors otherwise we need to develop a niche to win against competitors. 
        \item \textbf{average company size}: number of employees and other capital resources the competitors have on average. Ideally we have small competitors in our space
      \end{itemize}
  \end{itemize}
\end{document}
